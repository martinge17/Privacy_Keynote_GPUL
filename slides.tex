\documentclass{beamer}
\usepackage{graphicx}
\graphicspath{./Resources/}
\usepackage{hyperref}
\usepackage{qrcode} 

%Theme Choice:
%\usetheme{focus}
\usetheme{Warsaw}

%Title and options
\title{Ok Google! Deixa de espiarme}
\author{MartínG (Spaceduck)}
\date{29-03-2022}

\begin{document}

%Title 
\begin{frame}
  \titlepage
\end{frame}


\begin{frame}{Outline}
  \tableofcontents
\end{frame}


\section{Introduccion}
\subsection{Que é a Privacidade?}
\begin{frame}{Introducción}
  \framesubtitle{Que é?}
  \begin{block}{Privacidade}
    "La privacidad puede ser definida como el ámbito de la vida personal de un individuo, quien se desarrolla en un espacio reservado,
    el cual tiene como propósito principal mantenerse confidencial."
  \end{block}

\end{frame}

\subsection{Por que importa?}


\begin{frame}{Introducción}
  \framesubtitle{Por que importa?}

  \begin{block}{Privacidade}
    "Knowledge is power; Knowledge about you is power over you.
    Your information will be used to anticipate your actions
    and manipulate the way you shop, vote, and think."
  \end{block}\pause



  \begin{block}
    "Your personal information and private communications can be “cherrypicked” to make you look like
    a bad person or a criminal, even if you’re not."
  \end{block}

\end{frame}

\begin{frame}{Introducción}
  \framesubtitle{Por que importa?}

  Why Privacy Matters?:\quad
  \qrcode{https://whyprivacymatters.org/}
  \qquad

\end{frame}


\subsection{Thread Model}

\begin{frame}{Thread Model}
  \framesubtitle{Que é?}

  \begin{itemize}
    \item Protexer ao 100\% a túa privacidade é imposible. \pause
    \item Todo é un intercambio = Conveniencia x Privacidade/Seguridade \pause
    \item Necesitamos definir un plan.
  \end{itemize}

\end{frame}

\begin{frame}{Thread Model}
  \framesubtitle{Que é?}

  \begin{block}{Thread Model}
    Podémolo definir como unha lista coas ameazas para a nosa privacidade/seguridade que son  mais probable que ocorran
    e das que nos queremos defender.
  \end{block}\pause

\end{frame}

\section{Seguridade}
\subsection{Password Manager}

\begin{frame}{Seguridade}
  \framesubtitle{Password Manager}

  \begin{columns}
    \begin{column}{0.6\textwidth}
      % Your image included here
      \hspace{30pt}\includegraphics[scale=0.10]{Resources/bitwarden2.png}

      \vspace{1cm}

      \hspace{30pt}\qrcode{https://bitwarden.com/} %TODO: PENDIENTE DE ARREGLAR


    \end{column}

    \begin{column}{0.5\textwidth}
      \begin{block}{Bitwarden}
        \begin{itemize}
          \item Open Source
          \item E2E Encryption
          \item 2FA
          \item Selfhost (Opcional)
          \item Versión Gratuita e de Pago (10€ ano)
          \item Web,Android,Linux....
        \end{itemize}
      \end{block}

      \footnote{Outros Recomendados: KeePassXC}


    \end{column}

  \end{columns}

\end{frame}

\subsection{Disk Encryption}

\begin{frame}{Seguridade}
  \framesubtitle{Disk Encryption}

  \begin{columns}
    \begin{column}{0.5\textwidth}
      % Your image included here
      \includegraphics[scale=0.10]{Resources/VeraCrypt_Logo.png}

      \vspace{1cm}

      \qrcode{https://www.veracrypt.fr/en/Home.html} %TODO: PENDIENTE DE ARREGLAR


    \end{column}

    \begin{column}{0.5\textwidth}
      \begin{block}{VeraCrypt}
        \begin{itemize}
          \item Open Source
          \item File Containers (Folder Encryption)
          \item Filesystem Encryption
          \item Linux, MacOS, Windows
        \end{itemize}
      \end{block}
    \end{column}

  \end{columns}

\end{frame}



\section{Privacy}
\subsection{Providers}


%-----------------------------------------------------------------------------------

% TODO: Cambiar, quitar esa imaxen da publi e metela noutra diapo distinta

\begin{frame}{Search Engine}
  \framesubtitle{Problemas}

  \begin{columns}
    \begin{column}{0.5\textwidth}
      % Your image included here
      \includegraphics[scale=0.10]{Resources/gads-min.png}

    \end{column}

    \begin{column}{0.5\textwidth}
      \begin{block}{Google,Amazon, Bing, Yahoo!....}
        \begin{itemize}
          \item Rexistran toda a actividade
          \item Ligan esa actividade a unha conta / IP / Cookies.
          \item Usan esos datos para perseguirte con anuncios por toda a Web
          \item Comparten esos datos cos seus "socios".
        \end{itemize}
      \end{block}

    \end{column}

  \end{columns}

\end{frame}

%-----------------------------------------------------------------------------------

\begin{frame}{Search Engine}
  \framesubtitle{Solucións}

  \begin{columns}
    \begin{column}{0.5\textwidth}
      % Your image included here
      \includegraphics[scale=0.10]{Resources/duck.png}

      \vspace{1cm}

      \qrcode{https://duck.com} %TODO: PENDIENTE DE ARREGLAR

    \end{column}

    \begin{column}{0.5\textwidth}
      \begin{block}{DuckDuckGo}
        \begin{itemize}
          \item Alternativa mais completa
          \item Ten versión Onion e sen JavaScript
          \item "Semi" Open Source
          \item Usa a API de Bing (entre outros centos de fontes)
          \item Rexistras as busquedas pero non as IP
          \item Web,Android,Linux....
        \end{itemize}
      \end{block}

    \end{column}

  \end{columns}

\end{frame}

%-----------------------------------------------------------------------------------


\begin{frame}{Search Engine}
  \framesubtitle{Solucións}

  \begin{columns}
    \begin{column}{0.5\textwidth}
      % Your image included here
      \includegraphics[scale=0.10]{Resources/Startpage_logo.png}

      \vspace{1cm}

      \qrcode{https://startpage.com} %TODO: PENDIENTE DE ARREGLAR

    \end{column}

    \begin{column}{0.5\textwidth}
      \begin{block}{Startpage}
        \begin{itemize}
          \item Mesmos resultados que Google (Proxy)
          \item Solo rexistran metadatos como (SO, Navegador e Lengua)
          \item Web,Android,Linux....
        \end{itemize}
      \end{block}

    \end{column}

  \end{columns}

  \footnote{Outros Recomendados: Qwant}


\end{frame}

%-----------------------------------------------------------------------------------

\begin{frame}{BONUS!}

  \begin{columns}
    \begin{column}{0.5\textwidth}
      % Your image included here
      \includegraphics[scale=0.2]{Resources/Logo_searx_a.png}

      \vspace{1cm}

      \qrcode{https://searx.me} %TODO: PENDIENTE DE ARREGLAR

    \end{column}

    \begin{column}{0.5\textwidth}
      \begin{block}{Searx}
        \begin{itemize}
          \item Metasearch Engine
          \item Basado en instancias
        \end{itemize}
      \end{block}

    \end{column}

  \end{columns}

\end{frame}

%-----------------------------------------------------------------------------------

\begin{frame}{Email}
  \framesubtitle{Problemas}

  \begin{columns}

    \begin{column}{0.5\textwidth}
      \begin{block}{Gmail,Yahoo....}
        \begin{itemize}
          \item Rexistran toda a actividade (horas de envio, metadatos....)
          \item Escanean e clasifican todos os correos que recibes e envias, para "mellorar" a túa experiencia.
          \item Usan esos datos para perseguirte con anuncios por toda a Web
        \end{itemize}
      \end{block}

    \end{column}

  \end{columns}

\end{frame}

%-----------------------------------------------------------------------------------


\begin{frame}{Email}
  \framesubtitle{Problemas}

  \includegraphics[scale=0.3]{Resources/gmail_xd.png}

  \href{https://money.cnn.com/2017/06/23/technology/business/google-ad-scanning-email-stop/index.html}{Gmail Ads}

\end{frame}


\begin{frame}{Email}
  \framesubtitle{Problemas}

  \includegraphics[scale=0.3]{Resources/gmail.png}
  \href{https://www.wired.com/story/google-purchases-gmail-adobe-roundup/}{Gmail Purchases}

\end{frame}


%-----------------------------------------------------------------------------------


\begin{frame}{Email}
  \framesubtitle{Solucións}

  \begin{columns}
    \begin{column}{0.5\textwidth}
      % Your image included here
      \includegraphics[scale=0.5]{Resources/simple.png}

      \vspace{1cm}

      \qrcode{https://simplelogin.io} %TODO: PENDIENTE DE ARREGLAR

    \end{column}

    \begin{column}{0.5\textwidth}
      \begin{block}{Simplelogin}
        \begin{itemize}
          \item Email Alias
          \item Open Source
          \item Host on France
          \item Free (15 Alias), Paid (30E ano, Unlimited Alias)
        \end{itemize}
      \end{block}

    \end{column}

  \end{columns}

  \footnote{Outras opcións: AnonAddy}

\end{frame}

%-----------------------------------------------------------------------------------

%-----------------------------------------------------------------------------------


\begin{frame}{Email}
  \framesubtitle{Solucións}

  \begin{columns}
    \begin{column}{0.5\textwidth}
      % Your image included here
      \includegraphics[scale=0.2]{Resources/Tutanota_Logo.png}

      \vspace{1cm}

      \qrcode{https://tutanota.com/} %TODO: PENDIENTE DE ARREGLAR

    \end{column}

    \begin{column}{0.4\textwidth}
      \begin{block}{Tutanota}
        \begin{itemize}
          \item Germany Based
          \item E2E Encryption (Zero Access)
          \item Open Source
          \item Calendario e Contactos
          \item Plan Gratuito (1GB)
          \item Plan Premium (1€ mes)
          \item No IP log (Solo en certos casos *)
        \end{itemize}
      \end{block}

    \end{column}

  \end{columns}

  \footnote{We only log IP addresses of individual accounts in case of serious criminal acts such as murder, child pornography, robbery, bomb threats and blackmail after being served a valid court order by a German judge.}

\end{frame}

%-----------------------------------------------------------------------------------

\begin{frame}{Email}
  \framesubtitle{Problemas}

  \includegraphics[scale=0.4]{Resources/tutanota_mal.png}

  \href{https://www.cyberscoop.com/court-rules-encrypted-email-tutanota-monitor-messages/}{Tutanota Monitor}

\end{frame}

%-----------------------------------------------------------------------------------


\begin{frame}{Email}
  \framesubtitle{Solucións}

  \begin{columns}
    \begin{column}{0.5\textwidth}
      % Your image included here
      \includegraphics[scale=0.2]{Resources/Protonmail_logo.png}

      \vspace{1cm}

      \qrcode{https://protonmail.om} %TODO: PENDIENTE DE ARREGLAR

    \end{column}

    \begin{column}{0.5\textwidth}
      \begin{block}{Protonmail}
        \begin{itemize}
          \item Alternativa mais popular
          \item Open Source
          \item E2E Encryption (Zero Access)
          \item Encriptación a non usuarios de Proton (OpenPGP) + via Web
          \item Calendario e Contactos
          \item Cuenta gratuita e de pago (48€ ano)
          \item Solo baixo orden xudicial das cortes Suizas logean IP
        \end{itemize}
      \end{block}

    \end{column}

  \end{columns}

  \footnote{Outras opcións: CTemplar}

\end{frame}


\begin{frame}{Email}
  \framesubtitle{Recomendacións}
  \begin{itemize}
    \item Planificar
    \item Pouco a Pouco
    \item Recomendado ter varios emails
  \end{itemize}
\end{frame}


%-----------------------------------------------------------------------------------

%-----------------------------------------------------------------------------------

\begin{frame}{Drive}
  \framesubtitle{Problemas}

  \begin{columns}

    \begin{column}{0.5\textwidth}
      \begin{block}{Google Drive, Onedrive.....}
        \begin{itemize}
          \item Rexistran toda a actividade (metadatos....)
          \item Escanean e clasifican todos os arquivos e fotos subidas.
          \item Usan esos datos para mostrar publicidade.
        \end{itemize}
      \end{block}

    \end{column}

  \end{columns}

\end{frame}

%-----------------------------------------------------------------------------------


\begin{frame}{Drive}
  \framesubtitle{Solucións}

  \begin{columns}
    \begin{column}{0.5\textwidth}
      % Your image included here
      \includegraphics[scale=0.2]{Resources/Protonmail_logo.png}

      \vspace{1cm}

      \qrcode{https://protonmail.com} %TODO: PENDIENTE DE ARREGLAR

    \end{column}

    \begin{column}{0.4\textwidth}
      \begin{block}{ProtonDrive}
        \begin{itemize}
          \item O mesmo que ProtonMail
          \item Problema -> Precio
        \end{itemize}
      \end{block}

    \end{column}

  \end{columns}


\end{frame}


%-----------------------------------------------------------------------------------

\begin{frame}{Drive}
  \framesubtitle{Solucións}

  \begin{columns}
    \begin{column}{0.5\textwidth}
      % Your image included here
      \includegraphics[scale=0.2]{Resources/download.png}

      \vspace{1cm}

      \qrcode{https://nextcloud.com} %TODO: PENDIENTE DE ARREGLAR

    \end{column}

    \begin{column}{0.4\textwidth}
      \begin{block}{ProtonDrive}
        \begin{itemize}
          \item Alternativa a Suite Google Drive (Docs, Drive, Calendar ......)
          \item SelfHost ou Provider
          \item Alternativa moi Completa e recomendable
        \end{itemize}
      \end{block}

    \end{column}

  \end{columns}


\end{frame}

%-----------------------------------------------------------------------------------


\begin{frame}{Drive}
  \framesubtitle{Solucións}

  \begin{columns}
    \begin{column}{0.5\textwidth}
      % Your image included here
      \includegraphics[scale=0.1]{Resources/cryptomator.png}

      \vspace{1cm}

      \qrcode{https://cryptomator.org} %TODO: PENDIENTE DE ARREGLAR

    \end{column}

    \begin{column}{0.4\textwidth}
      \begin{block}{Cryptomator}
        \begin{itemize}
          \item Open Source
          \item Encripta automaticamente os arquivos
          \item Podese usar conxuntamente con calquera provedor (GDrive,Onedrive..)
          \item Recomendado se non queremos liarnos moito.
        \end{itemize}
      \end{block}

    \end{column}

  \end{columns}


  \footnote{Provedores recomendados: Tresorit, Sync.com, PCloud}


\end{frame}

%-----------------------------------------------------------------------------------


\begin{frame}{Docs}
  \framesubtitle{Solucións}

  \begin{columns}
    \begin{column}{0.5\textwidth}
      % Your image included here
      \includegraphics[scale=0.1]{Resources/cryptpad.png}

      \vspace{1cm}

      \qrcode{https://cryptpad.fr} %TODO: PENDIENTE DE ARREGLAR

    \end{column}

    \begin{column}{0.4\textwidth}
      \begin{block}{CryptPad}
        \begin{itemize}
          \item Open Source
          \item Conta con opción para selfhost
          \item E2E
          \item Usar sen rexistrarse
          \item Rexistro = Username + Password
          \item Word,Excel,Kanban,Whiteboard
          \item Free ou Pago (5GB = 5€, 75GB = 15€ /mes)
        \end{itemize}
      \end{block}

    \end{column}

  \end{columns}

\end{frame}


%-----------------------------------------------------------------------------------


\begin{frame}{Youtube}
  \framesubtitle{Problema}

  \begin{itemize}
    \item Publicidade Invasiva (Tracking)
    \item Perfilado
    \item Algoritmo (Mantente atrapado nun bucle)
    \item Acabas vendo o contido que Youtube quere que vexas.
  \end{itemize}

\end{frame}

%-----------------------------------------------------------------------------------

\begin{frame}{Youtube}
  \framesubtitle{Solucións}

  \begin{columns}

    \begin{column}{0.5\textwidth}
      % Your image included here

      \begin{figure}

        \includegraphics[scale=0.1]{Resources/lbry.png}
        \caption{Lbry}
        \label{Lbry}

      \end{figure}

      \vspace{0.4cm}

      \hspace{40pt}\qrcode{https://lbry.com} %TODO: PENDIENTE DE ARREGLAR

    \end{column}


    \begin{column}{0.5\textwidth}
      % Your image included here

      \begin{figure}

        \includegraphics[scale=0.8]{Resources/Logo_Textless.png}
        \caption{Odysee}
        \label{Odysee}

      \end{figure}

      \vspace{0.5cm}

      \hspace{40pt}\qrcode{https://odysee.com} %TODO: PENDIENTE DE ARREGLAR

    \end{column}

  \end{columns}

\end{frame}


%-----------------------------------------------------------------------------------


\begin{frame}{Youtube}
  \framesubtitle{Solucións: FrontEnds}

  \begin{columns}
    \begin{column}{0.5\textwidth}
      % Your image included here
      \includegraphics[scale=0.4]{Resources/piped.png}

      \vspace{1cm}

      \qrcode{https://github.com/TeamPiped/Piped/wiki/Instances} %TODO: PENDIENTE DE ARREGLAR

    \end{column}

    \begin{column}{0.4\textwidth}
      \begin{block}{Piped}
        \begin{itemize}
          \item Open Source
          \item No Ads
          \item No tracking
          \item Instancias
          \item SponsorBlock
          \item Non conectas cos servidores de Google
          \item Extensións para redirixir desde youtube.com
          \item Non recomendo rexistro
        \end{itemize}
      \end{block}

    \end{column}

  \end{columns}

\end{frame}

%-----------------------------------------------------------------------------------

\begin{frame}{Youtube}
  \framesubtitle{Solucións: FrontEnds}

  \begin{columns}
    \begin{column}{0.5\textwidth}
      % Your image included here
      \includegraphics[scale=0.1]{Resources/invidious-colored-vector.png}

      \vspace{1cm}

      \qrcode{https://invidious.io/} %TODO: PENDIENTE DE ARREGLAR

    \end{column}

    \begin{column}{0.4\textwidth}
      \begin{block}{Invidious}
        \begin{itemize}
          \item Open Source
          \item No Ads
          \item No tracking
          \item Instancias
          \item Extensións para redirixir desde youtube.com
          \item Non recomendo rexistro
        \end{itemize}
      \end{block}

    \end{column}

  \end{columns}

\end{frame}


%-----------------------------------------------------------------------------------

\begin{frame}{Youtube}
  \framesubtitle{Solucións: Apps}

  \begin{columns}
    \begin{column}{0.5\textwidth}
      % Your image included here
      \includegraphics[scale=0.4]{Resources/new_pipe_icon_5.png}

      \vspace{1cm}

      \qrcode{https://newpipe.net} %TODO: PENDIENTE DE ARREGLAR

    \end{column}

    \begin{column}{0.4\textwidth}
      \begin{block}{NewPipe (Android)}
        \begin{itemize}
          \item Free/Libre Software
          \item Non usa ningunha API de Google
          \item Non necesitas conta.
          \item Podes Importar Subscripcións + Feed
          \item No Ads nin tracking
          \item Pop-Up e Background
          \item Descargar videos e audio
          \item Todas as calidades

        \end{itemize}
      \end{block}

    \end{column}

  \end{columns}


  \footnote{Instalase en F-Droid}

\end{frame}

%-----------------------------------------------------------------------------------

\begin{frame}{DNS}
  \includegraphics[scale=0.2]{Resources/dns.png}

\end{frame}


\begin{frame}{DNS}
  \framesubtitle{Problema}

  \begin{block}{DNS}
    \begin{itemize}
      \item O teu provedor DNS sabe que sitios visitas.
      \item Por defecto usasa o da túa ISP ou Google
    \end{itemize}
  \end{block}

\end{frame}

%-----------------------------------------------------------------------------------

\begin{frame}{DNS}
  \framesubtitle{Solucións}

  \begin{columns}
    \begin{column}{0.5\textwidth}
      % Your image included here
      \includegraphics[scale=0.4]{Resources/Quad9_logo.png}

      \vspace{1cm}

      \qrcode{https://quad9.net} %TODO: PENDIENTE DE ARREGLAR

    \end{column}

    \begin{column}{0.4\textwidth}
      \begin{block}{Quad9}
        \begin{itemize}
          \item Non-Profit
          \item DoH, DoT, DNSCrypt, DNSSec
          \item Optional Filtering (Ads, Malware)
          \item No Logging
          \item Rapido
        \end{itemize}
      \end{block}

    \end{column}

  \end{columns}


\end{frame}

%-----------------------------------------------------------------------------------

\begin{frame}{DNS}
  \framesubtitle{Solucións}

  \begin{columns}
    \begin{column}{0.5\textwidth}
      % Your image included here
      \includegraphics[scale=0.4]{Resources/dnswatch.png}

      \vspace{1cm}

      \qrcode{https://dns.watch} %TODO: PENDIENTE DE ARREGLAR

    \end{column}

    \begin{column}{0.4\textwidth}
      \begin{block}{DNS.watch}
        \begin{itemize}
          \item DoH, DoT, DNSCrypt, DNSSec
          \item No Logging
          \item Rapido
        \end{itemize}
      \end{block}

    \end{column}

  \end{columns}


\end{frame}


%-----------------------------------------------------------------------------------
\subsection{Software}
%-----------------------------------------------------------------------------------

\begin{frame}{Navegador}
  \subtitle{Problema}

  \begin{block}{Chrome,Edge...}

    \begin{itemize}
      \item Recolecta metadatos de uso (conexión,tempo de uso...)
      \item Enlaza esos metadatos coa túa conta.
      \item Recolecta historial de busqueda
      \item Escasa protección cookies, publicidade, fingerprint..
    \end{itemize}

  \end{block}

\end{frame}

%-----------------------------------------------------------------------------------

\begin{frame}{Navegador}
  \subtitle{Problema}
  \includegraphics[scale=0.3]{Resources/chrome_cookie.png}

\end{frame}

\begin{frame}{Navegador}
  \subtitle{Problema}
  \includegraphics[scale=0.3]{Resources/chrome_incognito.png}

\end{frame}

\begin{frame}{Navegador}
  \subtitle{Problema}
  \includegraphics[scale=0.3]{Resources/marketshare.png}

\end{frame}

\begin{frame}{Navegador}
  \subtitle{Problema}

  Browser Privacy Test:\quad
  \qrcode{https://coveryourtracks.eff.org/}
  \qquad

\end{frame}



%-----------------------------------------------------------------------------------

\begin{frame}{Navegador}
  \framesubtitle{Solucións}

  \begin{columns}
    \begin{column}{0.5\textwidth}
      % Your image included here
      \includegraphics[scale=0.1]{Resources/firefox_logo.png}

      \vspace{1cm}

      Consellos Firefox:\quad
      \qrcode{https://www.privacyguides.org/browsers/} %TODO: PENDIENTE DE ARREGLAR

    \end{column}

    \begin{column}{0.4\textwidth}
      \begin{block}{Firefox}
        \begin{itemize}
          \item Open Source
          \item Enhanced Protection
          \item Block Tracking
          \item Recomendado instalar uBlock Origin
          \item Recomendado Cambiar Axustes
        \end{itemize}
      \end{block}

    \end{column}

  \end{columns}

\end{frame}


%-----------------------------------------------------------------------------------


%-----------------------------------------------------------------------------------

\begin{frame}{Navegador}
  \framesubtitle{Solucións}

  \begin{columns}
    \begin{column}{0.5\textwidth}
      % Your image included here
      \includegraphics[scale=0.5]{Resources/Brave_logo.png}

      \vspace{1cm}

      \qrcode{https://brave.com} %TODO: PENDIENTE DE ARREGLAR

    \end{column}

    \begin{column}{0.4\textwidth}
      \begin{block}{Brave}
        \begin{itemize}
          \item Open Source
          \item Bloquea Publicidade, Cookies, Trackers
          \item Recomendado desactivar todo o relacionado con Crypto
        \end{itemize}
      \end{block}

    \end{column}

  \end{columns}

\end{frame}


%-----------------------------------------------------------------------------------

\begin{frame}{Navegador}
  \framesubtitle{Solucións}

  \begin{columns}
    \begin{column}{0.5\textwidth}
      % Your image included here
      \includegraphics[scale=0.6]{Resources/UBlock_Origin.png}

      \vspace{1cm}

      \qrcode{https://ublockorigin.com/} %TODO: PENDIENTE DE ARREGLAR

    \end{column}

    \begin{column}{0.4\textwidth}
      \begin{block}{uBlock Origin}
        \begin{itemize}
          \item Open Source
          \item Bloquea Publicidade,Trackers....
          \item Consume poucos recursos
          \item Fai a navegación moi fluida.
        \end{itemize}
      \end{block}

    \end{column}

  \end{columns}

\end{frame}

%-----------------------------------------------------------------------------------

\begin{frame}{Navegador}
  \framesubtitle{Solucións}

  \begin{columns}
    \begin{column}{0.5\textwidth}
      % Your image included here
      \includegraphics[scale=0.22]{Resources/tosdr.png}

      \vspace{1cm}

      \qrcode{https://tosdr.org} %TODO: PENDIENTE DE ARREGLAR

    \end{column}

    \begin{column}{0.4\textwidth}
      \begin{block}{Terms of Service; Didn’t Read}
        \begin{itemize}
          \item Open Source
          \item Puntúa as webs segundo a súa Privacidade
          \item Pequeno resumo dos ToS
        \end{itemize}
      \end{block}

    \end{column}

  \end{columns}

\end{frame}

%-----------------------------------------------------------------------------------


\begin{frame}{Comunicacións}
  \subtitle{Problema}

  \begin{block}{MS Teams, Discord, Slack....}

    \begin{itemize}
      \item Recolectan datos de uso (conexión,tempo de uso...)
      \item Enlaza esos metadatos coa túa conta.
      \item Comparten datos con "socios".
      \item Non E2E.
    \end{itemize}

  \end{block}

\end{frame}

%-----------------------------------------------------------------------------------

\begin{frame}{Comunicacións}
  \framesubtitle{Solucións}

  \begin{columns}
    \begin{column}{0.5\textwidth}
      % Your image included here
      \includegraphics[scale=0.9]{Resources/element.png}

      \vspace{1cm}

      \qrcode{https://element.io/} %TODO: PENDIENTE DE ARREGLAR

    \end{column}

    \begin{column}{0.4\textwidth}
      \begin{block}{Element}
        \begin{itemize}
          \item Basado en Matrix
          \item Open Source
          \item Opción Self-Host
          \item Opción On-Premise
          \item Usado por moitas institucións públicas e universidades.
        \end{itemize}
      \end{block}

    \end{column}

  \end{columns}

  \footnote{Outras alternativas: Jitsi Meet...}

\end{frame}


%-----------------------------------------------------------------------------------



%-----------------------------------------------------------------------------------

\begin{frame}{Comunicacións}
  \subtitle{Problema}

  \begin{block}{Whatsapp,Telegram....}

    \begin{itemize}
      \item Recolectan metadatos(fechas, Contactos .....)
      \item Telegram non usa E2E por defecto (solo en algúns casos)
      \item Comparten datos con "socios".
      \item Non son FOSS
    \end{itemize}

  \end{block}

\end{frame}

%-----------------------------------------------------------------------------------

%-----------------------------------------------------------------------------------

\begin{frame}{Comunicacións}
  \framesubtitle{Solucións}

  \begin{columns}
    \begin{column}{0.5\textwidth}
      % Your image included here
      \includegraphics[scale=0.9]{Resources/signal.png}

      \vspace{1cm}

      \qrcode{https://signal.org} %TODO: PENDIENTE DE ARREGLAR

    \end{column}

    \begin{column}{0.4\textwidth}
      \begin{block}{Signal}
        \begin{itemize}
          \item Open Source
          \item Opción Self-Host
          \item Opción mais popular.
          \item Problemas: Necesitar número de telf.
        \end{itemize}
      \end{block}

    \end{column}

  \end{columns}

\end{frame}


%-----------------------------------------------------------------------------------
%-----------------------------------------------------------------------------------
%-----------------------------------------------------------------------------------

\subsection{Sistema Operativo}

%-----------------------------------------------------------------------------------
%-----------------------------------------------------------------------------------
%-----------------------------------------------------------------------------------


\begin{frame}{Sistema Operativo}
  \framesubtitle{Problemas}

  \begin{columns}
    \begin{column}{0.5\textwidth}
      % Your image included here
      \includegraphics[scale=0.2]{Resources/w10.png}

      \qrcode{https://www.theverge.com/2017/4/5/15188636/microsoft-windows-10-data-collection-documents-privacy-concerns}

    \end{column}

    \begin{column}{0.4\textwidth}
      \begin{block}{Windows}
        \begin{itemize}
          \item Require unha conta (Windows 11)
          \item Publicidade
          \item Recolección masiva de datos (uso,keylog....)
          \item Telemetria integrada na base do OS.
          \item Uso excesivo de recursos
          \item 100€ licencia
        \end{itemize}
      \end{block}

    \end{column}

  \end{columns}

\end{frame}




\begin{frame}{Sistema Operativo}
  \framesubtitle{Problemas}


  \begin{columns}
    \begin{column}{0.5\textwidth}
      % Your image included here
      \includegraphics[scale=0.5]{Resources/Tux.png}

      \vspace{1cm}

      \qrcode{https://distrochooser.de/es} %TODO: PENDIENTE DE ARREGLAR

    \end{column}

    \begin{column}{0.4\textwidth}
      \begin{block}{Linux,GNU + Linux, GNU/Linux..}
        \begin{itemize}
          \item Open Source
          \item Gran variedad de opcións a escoller.
          \item Completamente Personalizable.
          \item Adaptase as necesidades do usuario.
          \item Problemas: Gran variedad de opcións a escoller.
        \end{itemize}
      \end{block}

    \end{column}

  \end{columns}

\end{frame}


%------------------------------------------------------------------------------------------


\begin{frame}{Sistema Operativo}
  \framesubtitle{Problemas}

  \begin{columns}
    \begin{column}{0.5\textwidth}
      % Your image included here
      \includegraphics[scale=0.1]{Resources/android.png}

    \end{column}

    \begin{column}{0.4\textwidth}
      \begin{block}{Android}
        \begin{itemize}
          \item Open Source (AOSP)
          \item Comunidades de desarrollo moi activas. (ROM)
          \item Compatible con unha amplia gama de dispositivos.
          \item Telemetria integrada na base do OS.
          \item Lixeiro
          \item Problema??
        \end{itemize}
      \end{block}

    \end{column}

  \end{columns}

\end{frame}



\begin{frame}{Sistema Operativo}
  \framesubtitle{Problemas}

  \includegraphics[scale=0.12]{Resources/gapps.jpeg}

\end{frame}

\begin{frame}{Sistema Operativo}
  \framesubtitle{Problemas}

  \begin{columns}
    \begin{column}{0.5\textwidth}
      % Your image included here
      \includegraphics[scale=0.5]{Resources/gps.png}

    \end{column}

    \begin{column}{0.4\textwidth}
      \begin{block}{Google Play Services}
        \begin{itemize}
          \item Spyware
          \item Recolecta todo tipo de datos.
          \item Permite a Google acceder ao Smartphone
          \item Consume moitos recursos (Batería e Datos)
          \item Sin el moitas apps van a funcionar mal.
        \end{itemize}
      \end{block}

    \end{column}

  \end{columns}

  \footnote{Alternativas: MicroG ou simplemento AOSP}

\end{frame}


\begin{frame}
  \includegraphics[scale=0.4]{Resources/android_battery.png}
\end{frame}

\begin{frame}
  \includegraphics[scale=0.3]{Resources/location_google.png}
\end{frame}

\begin{frame}
  \includegraphics[scale=0.3]{Resources/assistant_messages.png}
\end{frame}

\begin{frame}

  Estudio:\quad
  \qrcode{https://www.scss.tcd.ie/Doug.Leith/Android_privacy_report.pdf}
  \qquad


\end{frame}


\begin{frame}

  \includegraphics[scale=0.5]{Resources/data _collection_summary.jpg}

\end{frame}

\begin{frame}

  \includegraphics[scale=0.4]{Resources/data2.png}

\end{frame}



\begin{frame}{Sistema Operativo}
  \framesubtitle{Solucións}

  \begin{columns}
    \begin{column}{0.5\textwidth}
      % Your image included here
      \includegraphics[scale=0.3]{Resources/fdroid.png}

      \qrcode{https://f-droid.org/}

    \end{column}

    \begin{column}{0.4\textwidth}
      \begin{block}{F-Droid}
        \begin{itemize}
          \item Tenda de Apps FOSS
          \item Fácil de usar.
        \end{itemize}
      \end{block}

    \end{column}

  \end{columns}

\end{frame}



\begin{frame}{Sistema Operativo}
  \framesubtitle{Solucións}

  FDroid Starter Pack :\quad
  \qrcode{https://www.reddit.com/r/fossdroid/comments/o1gmb8/the_im_new_to_fdroid_starter_pack/}

\end{frame}




\begin{frame}{Sistema Operativo}
  \framesubtitle{Solucións}

  \begin{columns}
    \begin{column}{0.5\textwidth}
      % Your image included here
      \includegraphics[scale=0.4]{Resources/blokada.png}

      \qrcode{https://f-droid.org/en/packages/org.blokada.alarm/}

    \end{column}

    \begin{column}{0.4\textwidth}
      \begin{block}{Blokada v4}
        \begin{itemize}
          \item Android Only
          \item Open Source
          \item Firewall.
          \item Permite bloquear moitos dos trackers en Android.
          \item Consume poucos recursos.
        \end{itemize}
      \end{block}

    \end{column}

  \end{columns}

  \footnote{Alternativa iOS: LockDown}

\end{frame}


\begin{frame}{Sistema Operativo}
  \framesubtitle{Solucións}

  \begin{columns}
    \begin{column}{0.5\textwidth}
      % Your image included here
      \includegraphics[scale=0.3]{Resources/lockdown.png}

      \qrcode{https://f-droid.org/en/packages/org.blokada.alarm/}

    \end{column}

    \begin{column}{0.4\textwidth}
      \begin{block}{LockDown}
        \begin{itemize}
          \item iOS e Mac
          \item Open Source
          \item Firewall.
        \end{itemize}
      \end{block}

    \end{column}

  \end{columns}

\end{frame}

\begin{frame}{Sistema Operativo}
  \framesubtitle{Solucións}

  \begin{columns}
    \begin{column}{0.5\textwidth}
      % Your image included here
      \includegraphics[scale=0.4]{Resources/osm.png}

      \qrcode{https://osmand.net/}

    \end{column}

    \begin{column}{0.4\textwidth}
      \begin{block}{OsmAnd}
        \begin{itemize}
          \item Open Source
          \item Permite mapas offline.
          \item Funciona moi ben.
          \item Desde F-Droid é gratuita. En GPlay é de pago
        \end{itemize}
      \end{block}

    \end{column}

  \end{columns}

\end{frame}

\subsection{IoT}

\begin{frame}
  \includegraphics[scale=0.3]{Resources/home.png}
\end{frame}


\begin{frame}{IoT}
  \framesubtitle{Solucións}

  \begin{columns}
    \begin{column}{0.5\textwidth}
      % Your image included here
      \includegraphics[scale=0.1]{Resources/pihole__.jpeg}

      \vspace{1cm}

      \qrcode{https://pi-hole.net/}

    \end{column}

    \begin{column}{0.4\textwidth}
      \begin{block}{Pi-Hole}
        \begin{itemize}
          \item Open Source
          \item Bloquea publicidade e trackers en toda a rede.
          \item Instalación moi sinxela (3 pasos) (30 minutos)
        \end{itemize}
      \end{block}

    \end{column}

  \end{columns}

\end{frame}


\begin{frame}{IoT}
  \framesubtitle{Solucións}

  \begin{columns}
    \begin{column}{0.5\textwidth}
      % Your image included here
      \includegraphics[scale=0.2]{Resources/hassio.jpeg}

      \vspace{1cm}


      \qrcode{https://www.home-assistant.io/}

    \end{column}

    \begin{column}{0.4\textwidth}
      \begin{block}{Home Assistant}
        \begin{itemize}
          \item Open Source
          \item Integración con moitos dispositivos
          \item Smart-Home en local.
        \end{itemize}
      \end{block}

    \end{column}

  \end{columns}

\end{frame}

\begin{frame}
  PrivacyGuides: \quad
  \qrcode{https://www.privacyguides.org/}

  \vspace{1cm}


  \hspace{30pt}reddit.com/r/degoogle \quad
  \hspace{30pt}\qrcode{https://www.reddit.com/r/degoogle/}

  \vspace{0.5cm}

  reddit.com/r/fossdroid \quad
  \qrcode{https://www.reddit.com/r/fossdroid/}

\end{frame}



\begin{frame}
  \frametitle{FIN}

  \begin{columns}
    \begin{column}{0.5\textwidth}
      % Your image included here
      \includegraphics[scale=0.3]{Resources/me2.png}

    \end{column}

    \begin{column}{0.4\textwidth}

      \LARGE Moitas Grazas!!

      \vspace{1cm}


      \qrcode{https://github.com/martinge17}

    \end{column}

  \end{columns}

\end{frame}




\end{document}
